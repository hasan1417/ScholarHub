\documentclass[12pt,a4paper]{article}

% Packages
\usepackage[utf8]{inputenc}
\usepackage[T1]{fontenc}
\usepackage{geometry}
\usepackage{graphicx}
\usepackage{hyperref}
\usepackage{booktabs}
\usepackage{tabularx}
\usepackage{listings}
\usepackage{xcolor}
\usepackage{float}
\usepackage{caption}
\usepackage{subcaption}
\usepackage{amsmath}
\usepackage{enumitem}

% Page geometry
\geometry{
    top=2.5cm,
    bottom=2.5cm,
    left=2.5cm,
    right=2.5cm
}

% Hyperref setup
\hypersetup{
    colorlinks=true,
    linkcolor=blue,
    filecolor=magenta,
    urlcolor=cyan,
    citecolor=blue
}

% Code listing style
\lstset{
    basicstyle=\ttfamily\small,
    breaklines=true,
    frame=single,
    backgroundcolor=\color{gray!10},
    keywordstyle=\color{blue},
    commentstyle=\color{green!50!black},
    stringstyle=\color{red!50!black}
}

\begin{document}

% Title Page
\begin{titlepage}
    \centering
    \vspace*{2cm}

    {\Huge\bfseries ScholarHub}\\[0.5cm]
    {\Large AI-Powered Research Collaboration Platform}\\[2cm]

    {\large\textbf{Progression Report}}\\[1cm]

    \vfill

    {\large
    \textbf{Submitted by:}\\[0.3cm]
    Hassan Ahmed Algizani\\
    Student ID: g202403940\\[0.8cm]
    \textbf{Supervised by:}\\[0.3cm]
    Dr. Irfan Ahmad\\[0.8cm]
    King Fahd University of Petroleum and Minerals (KFUPM)\\[1cm]
    }

    \vfill

    {\large December 2025}

\end{titlepage}

% Table of Contents
\tableofcontents
\newpage

% Abstract
\section*{Abstract}
\addcontentsline{toc}{section}{Abstract}

ScholarHub is a comprehensive, AI-powered research paper management and collaborative writing platform designed to streamline the academic research workflow. The platform addresses key challenges faced by researchers, including literature discovery, collaborative paper writing, reference management, and AI-assisted content generation.

This report presents the design, implementation, and features of ScholarHub, a full-stack web application built using modern technologies including React, FastAPI, PostgreSQL with vector embeddings, and real-time collaboration via CRDT-based synchronization. The platform integrates with multiple academic databases (ArXiv, PubMed, Semantic Scholar, etc.) for automated paper discovery and leverages OpenAI's language models for intelligent writing assistance.

Key contributions include a project-centric architecture that organizes research activities, real-time collaborative editing with conflict resolution, AI-powered literature review generation, and a multi-source academic paper discovery system with duplicate detection and relevance ranking.

\textbf{Keywords:} Research Collaboration, AI Writing Assistant, Literature Discovery, Real-time Collaboration, CRDT, Academic Platform

\newpage

% 1. Introduction
\section{Introduction}

\subsection{Background and Motivation}

Academic research has become increasingly collaborative and data-intensive. Researchers face numerous challenges in managing their workflow, including:

\begin{itemize}[leftmargin=*]
    \item \textbf{Literature Overload:} The exponential growth of published research makes it difficult to discover and track relevant papers.
    \item \textbf{Collaboration Barriers:} Teams working across institutions need effective tools for real-time collaboration on manuscripts.
    \item \textbf{Reference Management:} Organizing citations and generating bibliographies remains tedious and error-prone.
    \item \textbf{Writing Efficiency:} Academic writing is time-consuming and could benefit from AI assistance while maintaining scholarly rigor.
\end{itemize}

While various tools address individual aspects of research workflow (e.g., Zotero for references, Overleaf for LaTeX collaboration, Google Scholar for discovery), there exists a need for an integrated platform that combines these capabilities with modern AI assistance.

\subsection{Project Objectives}

ScholarHub aims to provide a unified platform that:

\begin{enumerate}[leftmargin=*]
    \item Enables project-based organization of research activities
    \item Supports real-time collaborative paper writing in LaTeX and rich-text formats
    \item Automates literature discovery from multiple academic databases
    \item Provides AI-powered writing assistance and literature review generation
    \item Facilitates team communication through integrated discussion channels
    \item Offers video conferencing with automatic transcription for research meetings
\end{enumerate}

\subsection{Report Organization}

The remainder of this report is organized as follows: Section 2 reviews related work and existing solutions. Section 3 presents the system architecture and technology stack. Section 4 details the implementation of core features. Section 5 demonstrates the platform through screenshots. Section 6 discusses challenges and lessons learned. Section 7 concludes with future work directions.

\newpage

% 2. Related Work
\section{Related Work and Background}

\subsection{Existing Research Tools}

Several tools currently address aspects of the research workflow:

\textbf{Reference Managers:} Tools like Zotero, Mendeley, and EndNote help researchers organize papers and generate citations. However, they primarily focus on reference management without integrated writing or discovery capabilities.

\textbf{Collaborative Writing:} Overleaf provides excellent LaTeX collaboration but lacks AI assistance and integrated reference discovery. Google Docs offers real-time collaboration but isn't optimized for academic writing.

\textbf{Literature Discovery:} Google Scholar, Semantic Scholar, and Connected Papers help discover related research, but require manual effort to integrate findings into projects.

\textbf{AI Writing Assistants:} Tools like Grammarly and ChatGPT offer writing assistance, but aren't designed for the specific needs of academic research with proper citations and domain knowledge.

\subsection{Technical Foundations}

\subsubsection{Real-time Collaboration}
Conflict-free Replicated Data Types (CRDTs) enable real-time collaborative editing without central coordination. Yjs, a JavaScript CRDT implementation, allows multiple users to edit documents simultaneously with automatic conflict resolution.

\subsubsection{Vector Embeddings for Search}
Modern semantic search uses neural network embeddings to represent text as high-dimensional vectors. PostgreSQL's pgvector extension enables efficient similarity search on these embeddings, powering intelligent paper recommendations.

\subsubsection{Large Language Models}
OpenAI's GPT models provide natural language understanding and generation capabilities that enable intelligent writing assistance, query enhancement, and literature summarization.

\newpage

% 3. System Architecture
\section{System Architecture}

\subsection{Overview}

ScholarHub follows a microservices-inspired architecture with clear separation between frontend, backend, and collaboration services. Figure~\ref{fig:architecture} illustrates the high-level system architecture.

\begin{figure}[H]
    \centering
    \fbox{\parbox{0.9\textwidth}{
        \centering
        \textbf{ScholarHub Architecture}\\[0.5cm]
        \begin{tabular}{c}
            \textbf{Frontend (React/Vite)} \\
            Port 3000 \\[0.3cm]
            $\downarrow$ REST API + WebSocket $\downarrow$ \\[0.3cm]
            \textbf{Backend (FastAPI)} \\
            Port 8000 \\[0.3cm]
            $\downarrow$ $\downarrow$ $\downarrow$ \\[0.3cm]
            \begin{tabular}{ccc}
                PostgreSQL & Redis & Collab Server \\
                (pgvector) & (Cache) & (Hocuspocus) \\
                Port 5432 & Port 6379 & Port 3001
            \end{tabular}
        \end{tabular}
    }}
    \caption{High-level System Architecture}
    \label{fig:architecture}
\end{figure}

\subsection{Technology Stack}

Table~\ref{tab:techstack} summarizes the technologies used in each layer of the application.

\begin{table}[H]
    \centering
    \caption{Technology Stack}
    \label{tab:techstack}
    \begin{tabularx}{\textwidth}{lXl}
        \toprule
        \textbf{Layer} & \textbf{Technology} & \textbf{Purpose} \\
        \midrule
        Frontend & React 18, TypeScript, Vite & User interface \\
        & TailwindCSS & Styling \\
        & TipTap / CodeMirror & Rich-text and LaTeX editors \\
        & React Query & Server state management \\
        \midrule
        Backend & FastAPI (Python) & REST API framework \\
        & SQLAlchemy 2.0 & ORM for database access \\
        & Pydantic & Data validation \\
        & OpenAI SDK & AI integration \\
        \midrule
        Database & PostgreSQL 15 & Primary data store \\
        & pgvector & Vector similarity search \\
        & Redis 7 & Caching and sessions \\
        \midrule
        Collaboration & Hocuspocus & Real-time sync server \\
        & Yjs & CRDT implementation \\
        \midrule
        Infrastructure & Docker Compose & Container orchestration \\
        & Daily.co & Video conferencing \\
        \bottomrule
    \end{tabularx}
\end{table}

\subsection{Database Schema}

The database consists of 26 models organized into logical groups:

\textbf{Core Entities:}
\begin{itemize}[leftmargin=*]
    \item \texttt{User} -- User accounts with authentication
    \item \texttt{Project} -- Research projects as organizational units
    \item \texttt{ResearchPaper} -- Papers within projects
    \item \texttt{Document} -- Uploaded PDFs and documents
    \item \texttt{Reference} -- Academic paper references
\end{itemize}

\textbf{Collaboration:}
\begin{itemize}[leftmargin=*]
    \item \texttt{ProjectMember} -- Project membership with roles
    \item \texttt{Comment} -- Threaded comments on content
    \item \texttt{CollaborationSession} -- Real-time editing sessions
\end{itemize}

\textbf{AI and Search:}
\begin{itemize}[leftmargin=*]
    \item \texttt{DocumentChunk} -- Chunked content with vector embeddings
    \item \texttt{AIChatSession} -- AI conversation history
    \item \texttt{AIArtifact} -- Generated content (summaries, reviews)
\end{itemize}

\newpage

% 4. Implementation
\section{Implementation and Features}

\subsection{Project Management}

ScholarHub uses a project-centric architecture where all research activities are organized within projects. Each project contains:

\begin{itemize}[leftmargin=*]
    \item Research papers (LaTeX or rich-text format)
    \item Reference library with approved citations
    \item Team members with role-based permissions (owner, editor, viewer)
    \item Discussion channels for team communication
    \item Sync space for video meetings
\end{itemize}

Projects support keywords and scope definitions that inform AI recommendations and literature discovery.

\subsection{Collaborative Paper Editor}

The platform supports two editor modes:

\textbf{LaTeX Editor:} Built on CodeMirror 6 with syntax highlighting, auto-completion, and real-time PDF preview. Supports collaborative editing via Yjs CRDT synchronization.

\textbf{Rich-Text Editor:} Built on TipTap (ProseMirror) with formatting tools, tables, images, and math equations via KaTeX. Also supports real-time collaboration.

Real-time synchronization is handled by a Hocuspocus server that:
\begin{enumerate}[leftmargin=*]
    \item Maintains document state using Yjs CRDTs
    \item Broadcasts changes to all connected clients
    \item Persists state to Redis for recovery
    \item Handles conflict resolution automatically
\end{enumerate}

\subsection{Literature Discovery}

The discovery system aggregates results from multiple academic APIs:

\begin{itemize}[leftmargin=*]
    \item ArXiv (preprints in physics, math, CS)
    \item PubMed (biomedical literature)
    \item Semantic Scholar (AI-powered academic search)
    \item OpenAlex (open scholarly metadata)
    \item CrossRef (DOI resolution and metadata)
    \item ScienceDirect (Elsevier publications)
\end{itemize}

Key features include:
\begin{itemize}[leftmargin=*]
    \item Smart query enhancement using GPT for better search terms
    \item Duplicate detection across sources
    \item Relevance ranking based on project keywords
    \item PDF access detection via Unpaywall
    \item Configurable source preferences per project
\end{itemize}

\subsection{AI Integration}

The platform integrates OpenAI's language models for several features:

\textbf{Writing Assistant:} Provides context-aware suggestions, helps rephrase text, and generates content based on project references.

\textbf{Literature Review Generation:} Synthesizes selected references into coherent literature review sections with proper citations.

\textbf{Chat Interface:} Answers questions about project content, explains concepts, and assists with research tasks.

\textbf{Query Enhancement:} Improves search queries for better literature discovery results.

\textbf{Meeting Transcription:} Uses Whisper API to transcribe recorded meetings and extract action items.

\subsection{Reference Management}

References are managed at both project and paper levels:

\begin{itemize}[leftmargin=*]
    \item Automatic metadata extraction from DOIs
    \item Citation parsing from various formats (BibTeX, RIS, etc.)
    \item Project reference library with approval workflow
    \item Paper-specific reference linking
    \item Bibliography generation in multiple styles
\end{itemize}

\subsection{Team Collaboration}

Beyond real-time editing, the platform supports:

\begin{itemize}[leftmargin=*]
    \item \textbf{Discussion Channels:} Organized conversation threads with file sharing
    \item \textbf{Comments:} Inline and section-level comments on papers
    \item \textbf{Notifications:} Activity feed for project updates
    \item \textbf{Sync Space:} Video meetings via Daily.co with recording and transcription
\end{itemize}

\newpage

% 5. Results and Screenshots
\section{System Demonstration}

This section presents screenshots of the ScholarHub platform demonstrating key features.

\subsection{Authentication}

\begin{figure}[H]
    \centering
    \includegraphics[width=0.85\textwidth]{screenshots/01_login_page.png}
    \caption{Login Page -- Secure authentication with email and password}
    \label{fig:login}
\end{figure}

The login page provides a clean, modern interface for user authentication. Users can register new accounts or sign in with existing credentials.

\newpage

\subsection{Projects Dashboard}

\begin{figure}[H]
    \centering
    \includegraphics[width=0.85\textwidth]{screenshots/02_projects_dashboard.png}
    \caption{Projects Dashboard -- Overview of all research projects}
    \label{fig:dashboard}
\end{figure}

The dashboard displays all projects accessible to the user, showing project titles, descriptions, and quick access to recent activity.

\subsection{Project Overview}

\begin{figure}[H]
    \centering
    \includegraphics[width=0.85\textwidth]{screenshots/03_project_overview.png}
    \caption{Project Overview -- Project details with team members and recent activity}
    \label{fig:overview}
\end{figure}

Each project has an overview page showing project description, keywords, team members with roles, and recent activity feed.

\newpage

\subsection{Papers Management}

\begin{figure}[H]
    \centering
    \includegraphics[width=0.85\textwidth]{screenshots/04_papers_list.png}
    \caption{Papers List -- All papers within a project}
    \label{fig:papers}
\end{figure}

The papers tab lists all research papers within a project, with options to create new papers using templates or from scratch.

\subsection{Paper Editor}

\begin{figure}[H]
    \centering
    \includegraphics[width=0.85\textwidth]{screenshots/05_paper_editor.png}
    \caption{Paper Editor -- Collaborative writing with AI assistance}
    \label{fig:editor}
\end{figure}

The paper editor supports both LaTeX and rich-text editing with real-time collaboration. The interface includes formatting tools, AI chat panel, and reference integration.

\newpage

\subsection{Research Library}

\begin{figure}[H]
    \centering
    \includegraphics[width=0.85\textwidth]{screenshots/07_library.png}
    \caption{Research Library -- Manage references and discover papers}
    \label{fig:library}
\end{figure}

The library provides a unified interface for managing references, viewing uploaded documents, and discovering new papers.

\subsection{Paper Discovery}

\begin{figure}[H]
    \centering
    \includegraphics[width=0.85\textwidth]{screenshots/08_discovery.png}
    \caption{Paper Discovery -- AI-powered literature search across multiple databases}
    \label{fig:discovery}
\end{figure}

The discovery feature searches multiple academic databases simultaneously, ranks results by relevance, and allows users to add papers directly to their project library.

\newpage

\subsection{Team Discussion}

\begin{figure}[H]
    \centering
    \includegraphics[width=0.85\textwidth]{screenshots/09_discussion.png}
    \caption{Discussion Channels -- Team communication with organized threads}
    \label{fig:discussion}
\end{figure}

Discussion channels enable team communication with support for file attachments, mentions, and organized conversation threads.

\subsection{Collaboration and Sync Space}

\begin{figure}[H]
    \centering
    \includegraphics[width=0.85\textwidth]{screenshots/10_collaborate.png}
    \caption{Collaboration -- Video meetings and real-time sync}
    \label{fig:collaborate}
\end{figure}

The collaboration tab provides access to video meetings with automatic transcription and action item extraction.

\newpage

\subsection{Reference Management}

\begin{figure}[H]
    \centering
    \includegraphics[width=0.85\textwidth]{screenshots/11_references.png}
    \caption{References -- Comprehensive reference management with metadata}
    \label{fig:references}
\end{figure}

The references view shows approved citations with full metadata, allowing easy management and export in various formats.

\newpage

% 6. Challenges and Lessons Learned
\section{Challenges and Lessons Learned}

\subsection{Technical Challenges}

\textbf{Real-time Collaboration:} Implementing conflict-free real-time editing required careful integration of Yjs CRDTs with both editor frameworks. Handling network disconnections and state recovery was particularly challenging.

\textbf{Multi-source Discovery:} Each academic API has different rate limits, response formats, and data quality. Building a unified interface required extensive normalization and error handling.

\textbf{Vector Search Performance:} As the document corpus grows, maintaining fast similarity search required careful indexing strategies with pgvector.

\textbf{AI Context Management:} Providing relevant context to AI models while staying within token limits required intelligent chunking and retrieval strategies.

\subsection{Lessons Learned}

\begin{enumerate}[leftmargin=*]
    \item \textbf{Start with Core Features:} Focus on essential functionality before adding advanced features.
    \item \textbf{API Abstractions:} Abstract external APIs behind unified interfaces to simplify maintenance.
    \item \textbf{Type Safety:} TypeScript on the frontend and Pydantic on the backend caught many errors early.
    \item \textbf{Feature Flags:} Implementing feature flags allowed gradual rollout of new capabilities.
\end{enumerate}

\newpage

% 7. Conclusion
\section{Conclusion and Future Work}

\subsection{Summary}

This report presented ScholarHub, an AI-powered research collaboration platform that integrates paper writing, literature discovery, reference management, and team collaboration into a unified experience. The platform demonstrates the potential of combining modern web technologies with AI capabilities to enhance research productivity.

Key achievements include:
\begin{itemize}[leftmargin=*]
    \item Project-centric architecture for organized research
    \item Real-time collaborative editing using CRDTs
    \item Multi-source literature discovery with intelligent ranking
    \item AI-powered writing assistance and literature review generation
    \item Integrated team communication and video meetings
\end{itemize}

\subsection{Future Work}

Several directions for future development include:

\begin{enumerate}[leftmargin=*]
    \item \textbf{Citation Network Analysis:} Visualize relationships between papers and identify influential works.
    \item \textbf{Automated Peer Review:} AI-assisted feedback on manuscript quality and completeness.
    \item \textbf{Mobile Application:} Native mobile apps for reading and annotation on the go.
    \item \textbf{Institutional Integration:} SSO and library access integration for universities.
    \item \textbf{Offline Support:} Enable offline editing with automatic synchronization.
    \item \textbf{Advanced Analytics:} Track reading progress, writing statistics, and team productivity.
\end{enumerate}

\subsection{Conclusion}

ScholarHub represents a comprehensive solution to the challenges of modern academic research. By combining collaborative writing, AI assistance, and literature management in a single platform, it has the potential to significantly improve research workflow efficiency. The modular architecture and modern technology stack provide a solid foundation for continued development and enhancement.

\newpage

% References
\section*{References}
\addcontentsline{toc}{section}{References}

\begin{enumerate}[leftmargin=*]
    \item Kleppmann, M., \& Beresford, A. R. (2017). A conflict-free replicated JSON datatype. \textit{IEEE Transactions on Parallel and Distributed Systems}, 28(10), 2733-2746.

    \item Brown, T., et al. (2020). Language models are few-shot learners. \textit{Advances in Neural Information Processing Systems}, 33, 1877-1901.

    \item Reimers, N., \& Gurevych, I. (2019). Sentence-BERT: Sentence embeddings using siamese BERT-networks. \textit{arXiv preprint arXiv:1908.10084}.

    \item FastAPI Documentation. (2024). FastAPI framework, high performance, easy to learn. \url{https://fastapi.tiangolo.com/}

    \item React Documentation. (2024). React -- A JavaScript library for building user interfaces. \url{https://react.dev/}

    \item Yjs Documentation. (2024). Shared data types for building collaborative software. \url{https://yjs.dev/}

    \item OpenAI. (2024). GPT-4 Technical Report. \url{https://openai.com/research/gpt-4}

    \item pgvector. (2024). Open-source vector similarity search for Postgres. \url{https://github.com/pgvector/pgvector}
\end{enumerate}

\end{document}
